\documentclass[11pt]{article}
\usepackage{graphicx}
\usepackage[bookmarks=true]{hyperref}
\usepackage{bookmark}
\usepackage{hyperref}
\usepackage{csquotes}
\usepackage{float}
\usepackage{wrapfig}
\usepackage[normalem]{ulem}
\useunder{\uline}{\ul}{}

\usepackage{array}
\newcolumntype{L}[1]{>{\raggedright\let\newline\\\arraybackslash\hspace{0pt}}m{#1}}
\newcolumntype{C}[1]{>{\centering\let\newline\\\arraybackslash\hspace{0pt}}m{#1}}
\newcolumntype{R}[1]{>{\raggedleft\let\newline\\\arraybackslash\hspace{0pt}}m{#1}}

\setlength{\parindent}{0pt}

\begin{document}

\begin{titlepage}
\begin{flushright}

\includegraphics[width=380px]{images/University_of_Pretoria_Logo.png}
\newline
\newline

\textbf {\LARGE Software Design Description} \newline

\textbf {\Large Linphone for Andriod Group Chat (Waterfall)}\newline

\textbf {\large User Interface Document}\newline

\centering \textbf {\large Authors:}

\begin{table}[H]
\large
\centering
\begin{tabular}{rl}
	Izak Blom & 13126777 \\
	David Breetzke & 12056503 \\
	Paul Engelke & 13093500 \\
	Prenolan Govender & 13102380 \\
	Jessica Lessev & 13049136 \\
\end{tabular}
\end{table}

\end{flushright}
\end{titlepage}

\setcounter{tocdepth}{3}
\setcounter{secnumdepth}{5}
\tableofcontents

\newpage
\section{Revision History}
\begin{table}[h]
\begin{tabular}{llll}
\textbf{Date}          & \textbf{Description}  & \textbf{Author}       & \textbf{Comments}   \\ \hline
\multicolumn{1}{|R{2cm}|}{26/08/2015} & \multicolumn{1}{L{4.5cm}|}{Document Creation} & \multicolumn{1}{l|}{Team Eclectic} & \multicolumn{1}{L{4cm}|}{Version 1} \\ \hline
\multicolumn{1}{|l|}{} & \multicolumn{1}{l|}{} & \multicolumn{1}{l|}{} & \multicolumn{1}{l|}{} \\ \hline
\multicolumn{1}{|l|}{} & \multicolumn{1}{l|}{} & \multicolumn{1}{l|}{} & \multicolumn{1}{l|}{} \\ \hline
\multicolumn{1}{|l|}{} & \multicolumn{1}{l|}{} & \multicolumn{1}{l|}{} & \multicolumn{1}{l|}{} \\ \hline
\multicolumn{1}{|l|}{} & \multicolumn{1}{l|}{} & \multicolumn{1}{l|}{} & \multicolumn{1}{l|}{} \\ \hline
\multicolumn{1}{|l|}{} & \multicolumn{1}{l|}{} & \multicolumn{1}{l|}{} & \multicolumn{1}{l|}{} \\ \hline
\multicolumn{1}{|l|}{} & \multicolumn{1}{l|}{} & \multicolumn{1}{l|}{} & \multicolumn{1}{l|}{} \\ \hline
\multicolumn{1}{|l|}{} & \multicolumn{1}{l|}{} & \multicolumn{1}{l|}{} & \multicolumn{1}{l|}{} \\ \hline
\multicolumn{1}{|l|}{} & \multicolumn{1}{l|}{} & \multicolumn{1}{l|}{} & \multicolumn{1}{l|}{} \\ \hline
\multicolumn{1}{|l|}{} & \multicolumn{1}{l|}{} & \multicolumn{1}{l|}{} & \multicolumn{1}{l|}{} \\ \hline
\multicolumn{1}{|l|}{} & \multicolumn{1}{l|}{} & \multicolumn{1}{l|}{} & \multicolumn{1}{l|}{} \\ \hline
\multicolumn{1}{|l|}{} & \multicolumn{1}{l|}{} & \multicolumn{1}{l|}{} & \multicolumn{1}{l|}{} \\ \hline
\multicolumn{1}{|l|}{} & \multicolumn{1}{l|}{} & \multicolumn{1}{l|}{} & \multicolumn{1}{l|}{} \\ \hline
\multicolumn{1}{|l|}{} & \multicolumn{1}{l|}{} & \multicolumn{1}{l|}{} & \multicolumn{1}{l|}{} \\ \hline
\multicolumn{1}{|l|}{} & \multicolumn{1}{l|}{} & \multicolumn{1}{l|}{} & \multicolumn{1}{l|}{} \\ \hline
\end{tabular}
\end{table}

\section{Document Approval}
\begin{table}[h]
\begin{tabular}{llll}
\textbf{Signature}     & \textbf{Printed Name} & \textbf{Title}        & \textbf{Comments}     \\ \hline
\multicolumn{1}{|l|}{} & \multicolumn{1}{L{3.5cm}|}{} & \multicolumn{1}{L{3.5cm}|}{} & \multicolumn{1}{L{4cm}|}{} \\ \hline
\multicolumn{1}{|l|}{} & \multicolumn{1}{l|}{} & \multicolumn{1}{l|}{} & \multicolumn{1}{l|}{} \\ \hline
\multicolumn{1}{|l|}{} & \multicolumn{1}{l|}{} & \multicolumn{1}{l|}{} & \multicolumn{1}{l|}{} \\ \hline
\multicolumn{1}{|l|}{} & \multicolumn{1}{l|}{} & \multicolumn{1}{l|}{} & \multicolumn{1}{l|}{} \\ \hline
\end{tabular}
\end{table}

\newpage

\section{Introduction}
\subsection{Purpose}
This document describes the plan for testing the development and functionality of the extended Linphone application software against the
user requirements as defined in the software requirements specification document [SRS]. The purpose of this acceptance test is to make sure
that the system developed during the Linphone extension project complies with the requirements listed in the SRS. These tests are executed in the Acceptance Test (AT) phase of the project.

\subsection{Scope}
The Linphone application is based on a distributed system. It ranges from a wide variety of technologies and incorporates a wide range of aspects. The extension to Linphone enables users to create, manage and participate in group chats with the side choice of using encryption while doing so. By hiding all complexity from the main server it makes the system easier to use, implement and test. Usability is also increased by providing an interactive user interface as a front-end for the users to interact with.

\subsection{Acronyms \& Abbreviations}
\begin{itemize}
\item \textbf{ATP} Acceptance Test Procedure
\item \textbf{AT} Acceptance Test
\item \textbf{SRS} Software Requirements Specification
\item \textbf{SDD} Software Design Description
\item \textbf{AES256} Advanced Encryption Standard (256-bit)
\end{itemize}
\subsection{Applicable Documents}
\subsubsection{External Documents}
\subsubsection{Internal Documents}
\begin{tabular}{ll}
\textbf{}       & \textbf{}  \\ \hline
\multicolumn{1}{|L{2cm}|}{SRS} & \multicolumn{1}{L{10cm}|}{Software Requirements Specification} \\ \hline
\multicolumn{1}{|L{2cm}|}{PSAC} & \multicolumn{1}{L{10cm}|}{Plan for Software Aspects of Certification} \\ \hline
\multicolumn{1}{|L{2cm}|}{SDD} & \multicolumn{1}{L{10cm}|}{Software Design Description} \\ \hline
\end{tabular}

\subsection{Test Environment}
Several aspects play a role in the testing environment. These include equipment used to test the product, technologies used, the environmental/surrounding conditions etc...\\
\textbf{Equipment}\\
The equipment used to test the extended functionality for the project will be as follows:
\begin{itemize}
\item Computer/Laptop with necessary IDE's and testing environment
\item Provided cellular phones with appropriate operating system and application installation
\end{itemize}
\textbf{Technologies}\\
The technologies used to test the functionality are mainly software oriented and include:
\begin{itemize}
\item JUnit for unit testing and integration testing
\item .............................
\end{itemize}

\textbf{Test Conditions}
Several factors describe the test conditions:
\begin{itemize}
\item Testing will take place in an integrated environment
\item The project is developed in a Linux environment using the Eclipse IDE. Thus testing will take place within the development paradigm using software to assist in unit and integration testing.
\item Since the project is a development for an android device, the bulk of the testing will occur on an android device. The specific device to be used will be the Zest cell phone.
\item All necessary actions are performed. In order to see the application in a certain state, we need to actually perform every step to get us there (clicking buttons, filling in text boxes, sending messages etc).
\item The only thing that is mocked during the testing is data such as message content etc
\end{itemize}



\subsection{Equipment Configuration \& Test Case Setup Instructions}


\section{Test Procedures \& Verification Results Record}
\subsection{Test Procedures Summary}
The following is a list of tests are to be performed:
\begin{enumerate}
\item Create group chat
\item 
\end{enumerate}
\subsection{Test 1: Create group chat}
\subsubsection{Description}
A user shall be able to create a new group.\\
\textbf{Preconditions:} For each of the pre-conditions below an exception is raised where that precondition is not met, to indicate service refusal.
The group will not be created if one of the following preconditions is not met:
\begin{itemize}
\item User must have access to the Linphone service.
\item The user may not create the group unless at least one other member is added.
\end{itemize}
\textbf{Postconditions:} The post-conditions specify the conditions which must hold true when the service has been provided.
\begin{itemize}
\item The creator is automatically assigned administrator rights.
\item An invite is sent to the other added users to ask them if they want to be part of the group.
\item The group is created with the respective profiles and members.
\item An error is displayed if the user attempts to create a group with less than two members in it.
\end{itemize}
\subsubsection{Result}


\subsection{Test 2}
\subsubsection{Description}
\subsubsection{Result}

\section{Summary Test Report}

\section{Traceability Matrix}
\end{document}