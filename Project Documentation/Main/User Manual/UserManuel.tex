\documentclass[11pt]{article}
\usepackage{graphicx}
\usepackage[bookmarks=true]{hyperref}
\usepackage{bookmark}
\usepackage{hyperref}
\usepackage{csquotes}
\usepackage{float}
\usepackage{wrapfig}
\usepackage{array}
\newcolumntype{L}[1]{>{\raggedright\let\newline\\\arraybackslash\hspace{0pt}}m{#1}}
\newcolumntype{C}[1]{>{\centering\let\newline\\\arraybackslash\hspace{0pt}}m{#1}}
\newcolumntype{R}[1]{>{\raggedleft\let\newline\\\arraybackslash\hspace{0pt}}m{#1}}

\setlength{\parindent}{0pt}

\begin{document}
\begin{titlepage}
\begin{flushright}

\includegraphics[width=380px]{images/University_of_Pretoria_Logo.png}
\newline
\newline

\textbf {\LARGE Software Design Description} \newline

\textbf {\Large Linphone for Andriod Group Chat (Waterfall)}\newline

\textbf {\large User Interface Document}\newline

\centering \textbf {\large Authors:}

\begin{table}[H]
\large
\centering
\begin{tabular}{rl}
	Izak Blom & 13126777 \\
	David Breetzke & 12056503 \\
	Paul Engelke & 13093500 \\
	Prenolan Govender & 13102380 \\
	Jessica Lessev & 13049136 \\
\end{tabular}
\end{table}

\end{flushright}
\end{titlepage}

\setcounter{tocdepth}{3}
\setcounter{secnumdepth}{5}
\tableofcontents
\newpage
\section{Revision History}
\begin{table}[h]
\begin{tabular}{llll}
\textbf{Date}          & \textbf{Description}  & \textbf{Author}       & \textbf{Comments}   \\ \hline
\multicolumn{1}{|R{2cm}|}{23/06/2015} & \multicolumn{1}{L{4.5cm}|}{Document Creation} & \multicolumn{1}{l|}{Team Eclectic} & \multicolumn{1}{L{4cm}|}{Version 1} \\ \hline
\multicolumn{1}{|R{2cm}|}{16/08/2015} & \multicolumn{1}{L{4.5cm}|}{Grammar Correction} & \multicolumn{1}{l|}{Team Eclectic} & \multicolumn{1}{L{4cm}|}{Version 1.01} \\ \hline
\multicolumn{1}{|R{2cm}|}{28/08/2015} & \multicolumn{1}{L{4.5cm}|}{Revision 1} & \multicolumn{1}{l|}{Team Eclectic} & \multicolumn{1}{L{4cm}|}{Version 2.0} \\ \hline
\multicolumn{1}{|l|}{} & \multicolumn{1}{l|}{} & \multicolumn{1}{l|}{} & \multicolumn{1}{l|}{} \\ \hline
\multicolumn{1}{|l|}{} & \multicolumn{1}{l|}{} & \multicolumn{1}{l|}{} & \multicolumn{1}{l|}{} \\ \hline
\multicolumn{1}{|l|}{} & \multicolumn{1}{l|}{} & \multicolumn{1}{l|}{} & \multicolumn{1}{l|}{} \\ \hline
\multicolumn{1}{|l|}{} & \multicolumn{1}{l|}{} & \multicolumn{1}{l|}{} & \multicolumn{1}{l|}{} \\ \hline
\multicolumn{1}{|l|}{} & \multicolumn{1}{l|}{} & \multicolumn{1}{l|}{} & \multicolumn{1}{l|}{} \\ \hline
\multicolumn{1}{|l|}{} & \multicolumn{1}{l|}{} & \multicolumn{1}{l|}{} & \multicolumn{1}{l|}{} \\ \hline
\multicolumn{1}{|l|}{} & \multicolumn{1}{l|}{} & \multicolumn{1}{l|}{} & \multicolumn{1}{l|}{} \\ \hline
\multicolumn{1}{|l|}{} & \multicolumn{1}{l|}{} & \multicolumn{1}{l|}{} & \multicolumn{1}{l|}{} \\ \hline
\multicolumn{1}{|l|}{} & \multicolumn{1}{l|}{} & \multicolumn{1}{l|}{} & \multicolumn{1}{l|}{} \\ \hline
\multicolumn{1}{|l|}{} & \multicolumn{1}{l|}{} & \multicolumn{1}{l|}{} & \multicolumn{1}{l|}{} \\ \hline
\multicolumn{1}{|l|}{} & \multicolumn{1}{l|}{} & \multicolumn{1}{l|}{} & \multicolumn{1}{l|}{} \\ \hline
\multicolumn{1}{|l|}{} & \multicolumn{1}{l|}{} & \multicolumn{1}{l|}{} & \multicolumn{1}{l|}{} \\ \hline
\end{tabular}
\end{table}

\section{Document Approval}
\begin{table}[h]
\begin{tabular}{llll}
\textbf{Signature}     & \textbf{Printed Name} & \textbf{Title}        & \textbf{Comments}     \\ \hline
\multicolumn{1}{|l|}{} & \multicolumn{1}{L{3.5cm}|}{} & \multicolumn{1}{L{3.5cm}|}{} & \multicolumn{1}{L{4cm}|}{} \\ \hline
\multicolumn{1}{|l|}{} & \multicolumn{1}{l|}{} & \multicolumn{1}{l|}{} & \multicolumn{1}{l|}{} \\ \hline
\multicolumn{1}{|l|}{} & \multicolumn{1}{l|}{} & \multicolumn{1}{l|}{} & \multicolumn{1}{l|}{} \\ \hline
\multicolumn{1}{|l|}{} & \multicolumn{1}{l|}{} & \multicolumn{1}{l|}{} & \multicolumn{1}{l|}{} \\ \hline
\end{tabular}
\end{table}

\newpage
\section{Introduction}
Linphone is the leading open source implementation of Voice over IP (VoIP) and Instant messaging functionalities, and is compatible with iOS, Android, Blackberry, Windows Phone, Windows desktop and web browser clients. This user manual will be focused specifically on the Android aspect. 
\section{Systems Overview}

\section{Systems Configuration}
Linphone has inside a separation between the user interfaces and the core engine, allowing to create various kinds of user interface on top of the same functionalities.\\

The user interface frontends:
\begin{itemize}
\item Gtk+ interface for windows, mac and linux
\item The console interface (linphonec, linphonecsh)
\item The iPhone application built in objective C
\item The Android application running in java
\item The Windows Phone application written in C\#
\end{itemize}
Liblinphone, the core engine: this is the library that implements all the functionalities of Linphone. \\
Liblinphone is a powerful SIP VoIP video SDK that anyone can use to add audio or video call capabilities to an application. It provides a high level api to initiate, receive, terminate audio and video calls.\\
Liblinphone relies on the following software components:
\begin{itemize}
\item Mediastreamer2, a powerful multimedia SDK to make audio/video streaming and processing.
\item oRTP, a simple RTP library.
\item belle-sip the SIP library.
\end{itemize}
Liblinphone and all its dependencies are written in pure C

\section{Installation}
To install the Linphone application simply follow the steps below:
\begin{enumerate}
\item Locate your Android device's Play store
\item Under the search tab, search for "Linphone"
\item Locate the Linphone application and click on it \includegraphics[width=10px]{./images/icon.jpg}
\item Follow you device's guidline to install the application
\item Once installed, launch the application by clicking on the Linphone icon in your applications list on you device
\item Follow the prompts further to register if you are a new user or to log in if you already have an account.
\end{enumerate}

\section{Getting Started}
In order to make use of the application one needs to have a linphone.org account and SIP account.Follow the steps bellow to create one:
\begin{enumerate}
\item Launch the application
\item Click the "Start" button located on the bottom right of the screen
\item Click on "Create an account on linphone.org" if a new user 
\subitem Follow the prompts on the screen to further creat an account
\item Click on "I have already a linphone.org account" if one already has an account
\subitem Enter the neccessary account information and click "Sign in"
\item Click on "I already have a SIP account" if one already has a SIP address to use
\subitem Enter the neccessary account information and click "Sign in"
\item Once logged in you will be presented with the application user interface
\end{enumerate}

\section{Using The System}
\subsection*{Using the Group Chat functionality}
Below is a diagram showing how one can navigate through the interfaces involved in the group chat functionality. \\
\includegraphics[width=300px]{./images/flow.png}

When on the "Chat" user interface, one will be presented with the following options on the top panel of the screen:
\begin{itemize}
\item New Conversation
\item New Group Chat
\item Edit
\end{itemize}

\subsection{Guidelines}
Below are guidelines on how to use the system. Guidelines are provided for various sections. The sections covered by the guidelines are as follows:
\begin{itemize}
\item Locate list of group chats
\item Create new group chat.
\item Chat in the group chat
\item View group chat information
\item Edit existing group chat.
\item Add/remove members from existing group chat.
\item Green Path - from the flow diagram.
\item Orange Path - from the flow diagram. 
\end{itemize}

\subsubsection*{Locate list of group chats}
Follow the steps below to locate the location/list of all your current group chats:
\begin{enumerate}
\item Start the application
\item On the "Home" screen click the "Chat" option as shown in the diagram\\
\includegraphics[width=50px]{images/mainScreen.png}
\item Click on the "Groups" button\\
\includegraphics[width=50px]{images/ChatlistNav.png}
\item You are now presented with the list of group chats
\begin{enumerate}
\item If a group chat from the list is selected, the group chat will open\\
\end{enumerate}
\includegraphics[width=50px]{images/Grouplist.png}
\item to navigate back to your private chat lists, click on the "Chats" button\\
\includegraphics[width=50px]{images/GrouplistNav.png}
\end{enumerate}

\subsubsection*{Create New Group Chat}
Follow the steps below to create an launch a new group chat:
\begin{enumerate}
\item Start the application
\item On the "Home" screen click the "Chat" option as shown in the diagram\\
\includegraphics[width=50px]{images/mainScreen.png}
\item Click on the "New Group Chat" button\\
\includegraphics[width=50px]{images/ChatlistCG.png}
\item You are now presented with a user interface where necessary information is needed in order to create the group chat
\begin{enumerate}
\item Enter a desired group name
\item Choose the type of encryption you wish to use
\item Add members to participate in the group chat - Note: at least two members must be added
\subitem Type the member's SIP address in the text edit
\subitem Click on the green "+" button
\subitem To clear the text edit click on the red "X" to the left of the text edit
\subitem To remove a added member click on the red "X" next to the corresponding member you wish to remove
\item Click "Next"
\end{enumerate}
\includegraphics[width=50px]{images/GroupChatCreation.png}
\item You will now be presented with the actual group chat where you can proceed to chat with the members\\
\includegraphics[width=50px]{images/groupchat.png}
\end{enumerate}

\subsubsection*{Chat in Group Chat}
Follow the steps below to participate in a group chat in which you are a member:
\begin{enumerate}
\item Start the application
\item On the "Home" screen click the "Chat" option as shown in the diagram\\
\includegraphics[width=50px]{images/mainScreen.png}
\item Click on the "Groups" button\\
\includegraphics[width=50px]{images/ChatlistNav.png}
\item You are now presented with a list of the group chats in which you are a member and may participate
\item Click on the group chat in which you want to "chat"/participate\\
\includegraphics[width=50px]{images/Grouplist.png}
\item You will be presented with the group chat interface\\
\includegraphics[width=50px]{images/groupchat.png}
\item To type a message:
\begin{enumerate}
\item Click on the text edit\\
\includegraphics[width=50px]{images/groupchatEdit.png}
\item The keyboard will pop-up. Type the desired message
\item Click the "Send" button to send the message to the group and all the members in the group\\
\includegraphics[width=50px]{images/groupchatSend.png}
\end{enumerate}
\item To send a picture/image to the group:
\begin{enumerate}
\item Click on the "Pic" button\\
\includegraphics[width=50px]{images/groupchatSendPic.png}
\item You will be asked whether to send a picture already stored in you phone gallery or to take a new picture. Click on the button that corresponds to your desired choice.
\item Follow the respective steps for the picture option and select the image you wish to send
\item Once you locate the picture you wish to send, click on it and it will automatically send the the rest of the group members
\end{enumerate}
\end{enumerate}

\subsubsection*{View Group Chat Information}
Follow the steps below to view a group chat's details and information:
\begin{enumerate}
\item Start the application
\item On the "Home" screen click the "Chat" option as shown in the diagram\\
\includegraphics[width=50px]{images/mainScreen.png}
\item Click on the "Groups" button\\
\includegraphics[width=50px]{images/ChatlistNav.png}
\item You are now presented with a list of the group chats
\item Click on the group chat for which you wish to see details\\
\includegraphics[width=50px]{images/Grouplist.png}
\item The actual group chat will open
\item Click on the "Group Information" button on the top right of the screen\\
\includegraphics[width=50px]{images/groupchatInfo.png}
\item The groups information will be displayed
\end{enumerate}

%Looking at the diagram above, the steps to navigate through the user interface will be listed:
%\textbf{Green Path}
%\begin{enumerate}
%\item Once logged in you will be presented with the application user interface
%\item Click on the "Chat" menu item located on the bottom panel
%\item Click "New Group Chat" option located on the top panel
%\item You will be presented with a screen to enter necessary information to create a new group chat
%\subitem Enter a desired group name
%\subitem Choose the type of encryption you wish to use
%\subitem Add members to participate in the group chat - Note: at least two members must be added
%\subitem Click "Next"
%\item You will now be presented with the actual group chat
%\item You can now proceed to "chat" in the group chat or view the group information by clicking on "Group %Info" in the group chat
%\subitem You can edit the information by clicking "Edit" and changing the information desired.
%\end{enumerate}


\section{Troubleshooting}

\section{Appendix B}
\listoffigures

\end{document}
